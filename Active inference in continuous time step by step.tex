\documentclass[a4paper, 10pt]{article}

\usepackage[T1]{fontenc} % To choose the font encoding of the output text. You might need it if you are writing documents in a language other than English.

\usepackage[utf8]{inputenc} % To choose the encoding of the input text. You might need it if you are writing documents in a language other than English (foundamental in particular for letters with accent mark).

\usepackage[english, italian]{babel} % It provides the internationalization of LaTeX. It has to be loaded in any document, and you have to give as an option the main language you are going to use in the document.

\usepackage[a4paper, total={6in, 8in}]{geometry} % To modify margins

\newcommand{\virgolette}[1]{``#1''} % New function to use correct open and close inverted commas

\usepackage[autostyle]{csquotes} % Package necessary to biblatex
\usepackage[sorting=nyt, style=authoryear]{biblatex} % Advanced bibliography handling. It is the package to use for writing a thesis. bibstyle=authortitle, backend=biber
\addbibresource{activeinference.bib} % File from which take the bibliography 

\usepackage{amsmath,amsfonts,amsthm,bm} % Math packages

\setlength{\parskip}{0.5em}


\title{Active Inference in continuous time notes}
\author{Federico Maggiore}

\begin{document}
\selectlanguage{english}

\maketitle

\paragraph{\textbf{Summary notation}}
\begin{itemize}

\item $\mathbf x = \lbrace{ x_i \rbrace}_{i=1}^{D}$ environmental variables of the $D$-dimensional space constituting latent or hidden states;

\item $\mathbf{s} = \lbrace{ s_i \rbrace}_{i=1}^{S}$ body sensors input;

\item $P(\mathbf{s},\mathbf{x})$ \emph{G-density};

\item $P(\mathbf{x}|\mathbf{s})$ \emph{Posterior}; 

\item $P(\mathbf{s}|\mathbf{x})$ \emph{Likelihood};

\item $P(\mathbf x)$ \emph{Prior};

\item $P(\mathbf s)=\int P(\mathbf s|\mathbf x)P(\mathbf x) d\mathbf x $ \emph{marginal likelihood}

\item $Q(\mathbf x)$ \emph{R-density}

\item $F \equiv \int Q(\mathbf x) \ln \frac{Q(\mathbf x)}{P(\mathbf x,\mathbf s)}d\mathbf x $ \emph{Variational Free Energy}

\end{itemize}

\section*{Free Energy Principle (FEP)}
The goal of an agent is to determine the probability of the hidden states given some sensory inputs:

\begin{equation}
P(\mathbf{x}|\mathbf{s}) = \frac{P(\mathbf{s},\mathbf{x})}{P(\mathbf{s})} = \frac{P(\mathbf{s}|\mathbf{x})P(\mathbf{x})}{P(\mathbf{s})}
\end{equation}
with
\begin{itemize}

\item $P(\mathbf{s},\mathbf{x})$ \emph{G-density}, beliefs about the states assumed to be encoded by the agent;

\item $P(\mathbf{x}|\mathbf{s})$ \emph{Posterior}, i.e. probability of hidden causes $x$ given observed sensory data; 

\item $P(\mathbf{s}|\mathbf{x})$ \emph{Likelihood}, i.e. organism's assumptions about sensory input $\mathbf{s}$ given the hidden causes $\mathbf{x}$;

\item $P(\mathbf x)$ \emph{Prior}, i.e. agent's beliefs about hidden causes \textbf{before} that $\mathbf s$ are received;

\item $P(\mathbf s)=\int P(\mathbf s|\mathbf x)P(\mathbf x) d\mathbf x $ \emph{marginal likelihood}, i.e. normalization factor.

\end{itemize}

For the agent it's not necessary to compute the complete posterior distribution, it has only to find the hidden state -or at least a good approximation- that maximize the posterior, i.e. $\arg \max_{\mathbf{x}} P(\mathbf x|\mathbf s)$.
The problem with the exact Bayesian scheme, is that $P(\mathbf s)$ is often impossible to calculate, and moreover $P(\mathbf x|\mathbf s)$ may not take a standard shape and could not have a summary statistics. 

A biologically plausible technique consist in using an auxiliary distribution $Q(\mathbf x)$ called \emph{recognition density} (\emph{R-density}) that has to be optimized to became a good approximation of the posterior.  

In order to do this the Kullback-Leibler  divergence is minimized:
\begin{equation}
\begin{split}
D_{KL} (\, Q(\mathbf x)\, ||\, P(\mathbf x|\mathbf s)\, ) & = \int Q(\mathbf x) \ln \frac{Q(\mathbf x)}{P(\mathbf x|\mathbf s)} d\mathbf x \\
                                  & = \int Q(\mathbf x) \ln \frac{Q(\mathbf x)P(\mathbf s)}{P(\mathbf x,\mathbf s)} d\mathbf x \\
                                  & = \int Q(\mathbf x) \ln \frac{Q(\mathbf x)}{P(\mathbf x,\mathbf s)} d\mathbf x + \ln P(\mathbf s) \int Q(\mathbf x) d\mathbf x \\
                                  & = F + \ln P(\mathbf s)
\end{split}
\end{equation}

where
\begin{itemize}

\item $F \equiv \int Q(\mathbf x) \ln \frac{Q(\mathbf x)}{P(\mathbf x,\mathbf s)}d\mathbf x = - \int Q(\mathbf x) \ln P(\mathbf x,\mathbf s) d\mathbf x + \int Q(\mathbf x) \ln Q(\mathbf x) d\mathbf x $ is the \emph{Variational Free Energy}, a quantity that depends on the R-density and the knowledge about the environment i.e. the G-density $P(\mathbf s, \mathbf x) = P(\mathbf s|\mathbf x)P(\mathbf x)$ that we are assuming the agent has. 
\item $\ln P(\mathbf s)$ is a term independent of the recognition density $Q(\mathbf x)$ ( $\Rightarrow$ minimizing F with respect to $Q(\mathbf x)$ will minimize the $D_{KL}$ )

\end{itemize}

\subsection*{Laplace approximation}

Often optimizing $F$ for arbitrary $Q(\mathbf x)$ is particularly complex. Moreover, it is assumed that neural activity parametrise sufficient statistic.
For these reasons, a common approximation is to assume that the R-density take a Gaussian form. 

For now the D=1 case in detail, leaving the formulation of the multivariate case for later.

Given a probability density $Q(x)$ with a peak at point $\mu$, the Taylor-expansion of the logarithm around his peak is
\begin{equation}
\ln Q(x) \simeq \ln Q(\mu) - \frac{1}{2} \frac{(x-\mu)^2}{\Sigma} 
\end{equation}
with
\begin{equation}
\frac{1}{\Sigma} = - \frac{\partial^{2} }{\partial x^2} \ln P(x) \bigg\rvert_{x=\mu} 
\end{equation}
Now it is possible to approximate the probability distribution $Q(x)$ with the distribution
\begin{equation}
\mathcal{N}(x;\mu, \Sigma) = \frac{1}{\sqrt{ 2 \pi \Sigma}} \, e^{\frac{(x-\mu)^2}{2 \Sigma}}
\end{equation}
a Gaussian distribution that has been normalized using the factor
\begin{equation}
Z_{\mathcal{N}} = Q(\mu) \sqrt{2 \pi \Sigma } 
\end{equation}

Generalizing for a density $Q(\mathbf x)$ over a K-dimensional space $\mathbf x$ with peak at $\boldsymbol \mu$
\begin{equation}
\ln Q(\mathbf x) \simeq \ln Q(\boldsymbol{\mu}) - \frac{1}{2} (\boldsymbol x-\boldsymbol \mu)^T \boldsymbol \Sigma^{-1} (\boldsymbol x-\boldsymbol \mu)
\end{equation}
with
\begin{equation}
\left[ \boldsymbol \Sigma^{-1}\right]_{i,j} = - \frac{\partial^{2} }{\partial x_i \partial x_j} \ln P(\boldsymbol x) \bigg\rvert_{\mathbf x=\boldsymbol \mu} 
\end{equation}
is possible to approximate $Q(\mathbf x)$ with the multivariate Gaussian distribution
\begin{equation}
\mathcal{N}(\mathbf x;\boldsymbol \mu, \boldsymbol \Sigma) = \frac{1}{\sqrt{( 2 \pi)^{K} \det \boldsymbol \Sigma}} \, e^{\frac{1}{2} (\boldsymbol x-\boldsymbol \mu)^T \boldsymbol \Sigma^{-1} (\boldsymbol x-\boldsymbol \mu)}
\end{equation}

This approximation is particularly useful to approximate integrals. 






\end{document}


